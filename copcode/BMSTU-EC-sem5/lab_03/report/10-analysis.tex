\chapter*{Основные теоретические сведения}
\addcontentsline{toc}{chapter}{Основные теоретические сведения}
Программа PCLAB предназначена для исследования производительности x86 совместимых ЭВМ c IA32 архитектурой, работающих под управлением операционной системы Windows (версий 95 и старше). Исследование организации ЭВМ заключается в проведении ряда экспериментов, направленных на построение зависимостей времени обработки критических участков кода от изменяемых параметров. Набор реализуемых программой экспериментов позволяет исследовать особенности построения современных подсистем памяти ЭВМ и процессорных устройств, выявить конструктивные параметры конкретных моделей ЭВМ. 

Процесс сбора и анализа экспериментальных данных в PCLAB основан на процедуре профилировки критического кода, т.е. в измерении времени его обработки центральным процессорным устройством. При исследовании конвейерных суперскалярных процессорных устройств, таких как 32-х разрядные процессоры фирмы Intel или AMD, способных выполнять переупорядоченную обработку последовательности команд программы, требуется использовать специальные средства измерения временных интервалов и запрещения переупорядочивания микрокоманд. 